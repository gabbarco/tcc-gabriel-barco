% !TeX root = ../artigo.tex
\begin{singlespace}
\textbf{Resumo: }
O OpenLDAP é um serviço de diretório de código aberto, que permite a autenticação e gerenciamento de usuários em uma rede, este software pode ser instalado junto a um servidor para gerenciar informações sobre usuários, grupos, permissões e outros recursos encontrados nele. O objetivo deste trabalho é instalar e configurar este sistema como uma alternativa ao Active Directory da Microsoft (que envolve maiores custos para instalação por conta de sua licença), e também administrar usuários e outros recursos que a ferramenta possibilita. Além disso busca-se por meio desse trabalho estudar a viabilidade desse software em uma rede de pequeno porte, analisando sua escalabilidade e flexibilidade. A metodologia empregada consistirá na instalação e configuração do OpenLDAP em uma máquina virtual Debian, hospedada em um ambiente Proxmox. O propósito é autenticar e gerenciar as contas de usuário dos membros do "PET Computação IFTM Campus Ituiutaba", proporcionando facilidade de uso e disponibilizando recursos das máquinas de forma eficiente por meio desta ferramenta.\\
\textbf{Palavras-chave: }
OpenLDAP. Autenticação de Usuários. Serviços de diretório.
\end{singlespace}
