% !TeX root = ../artigo.tex
\section{CONCLUSÃO}
Com base nos resultados obtidos anteriormente, pode-se afirmar que o OpenLDAP é uma solução viável e econômica para gerenciamento de usuários em redes de pequeno porte, capaz de oferecer recursos robustos para a gestão de grupos e permissões. Sua natureza de código aberto, não apenas reduz custos, mas também oferece uma comunidade ativa de desenvolvedores e usuários, proporcionando suporte contínuo e atualizações frequentes.

Além disso, a facilidade de instalação e configuração, conforme detalhado neste artigo, não apenas simplifica o processo para os administradores de rede, mas também reduz potenciais erros durante a implementação.

Assim, é possível concluir que o OpenLDAP não apenas atende, mas excede as expectativas para as finalidades adotadas neste projeto, consolidando-se como uma escolha sólida para organizações que buscam eficiência, economia e segurança em seus ambientes de rede.
