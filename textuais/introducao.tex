% !TeX root = ../artigo.tex
\section{INTRODUÇÃO}
\pagenumbering{arabic}
Os serviços de diretório têm sido cada vez mais utilizados em todas as
organizações do mundo, sejam faculdades, empresas, entre outras. Essa
grande demanda se deve ao fato de que em qualquer uma dessas é praticamente essencial centralizar a autenticação de usuários, com o objetivo de reduzir custos e facilitar a administração \cite{berbelini}. Com esse aumento em sua necessidade, vários serviços como Active Directory (AD), OpenLDAP e Edirectory, vêm se destacando no mercado.

Estas ferramentas não se limitam apenas à autenticação de usuários, como também podem servir como suporte para a realização de atividades tais como nomeação, localização, segurança, entre outras relacionadas a gerir a infraestrutura de recursos nas organizações \cite{cruz2023}.

Esses serviços surgiram, mediante à necessidade de se ter um único diretório que possibilitasse ao usuário acessar todos os recursos da empresa com apenas uma única senha, o que não era antes feito, assim, funcionários de uma empresa por exemplo, possuíam várias delas para acessarem diferentes setores. 

Como a própria Microsoft define o seu próprio sistema, "o  Active  Directory é  um  serviço  de  diretório  que  armazena  informações sobre objetos em rede e disponibiliza essas informações a usuários e administradores de rede" \cite{microsoftlearn}.

Com a sua popularização, tornaram-se indispensáveis para qualquer organização. Logo, não demorou para que fossem criados serviços de código aberto, isto é, acessíveis a qualquer pessoa que queira examinar, modificar, aprimorar, distribuir ou usá-los para qualquer finalidade. Isso se refere ao caso do OpenLDAP, que, como outros destes serviços, é mantido pela própria comunidade de desenvolvedores que o utilizam.

"Comumente o AD é confundido como o concorrente do LDAP, o que não é o caso, porquê ele não é nada mais que um exemplo de um serviço de diretório que suporta LDAP" \cite{bertolli}. O LDAP pode ser definido então, como o protocolo dos sistemas de diretório, ou seja, um meio para consultar os itens em qualquer um dos mesmos que o suportam.