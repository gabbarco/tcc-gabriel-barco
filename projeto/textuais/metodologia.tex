% !TeX root = ../projeto.tex
\newpage
%\thispagestyle{empty}
%\pagenumbering{arabic}

\section{METODOLOGIA}
\label{sec:metodologia}
O desenvolvimento do projeto proposto, demanda o estudo e desenvolvimento de um gerador de tráfego para o protocolo HTTP. Dessa forma, o projeto demanda o uso intenso de ferramentas para o desenvolvimento de programas computacionais.

Com relação ao hardware, é recomendado que o desenvolvedor possua um computador com processador core Intel i5 ou i7, ou seus respectivos competidores de outros fabricantes. A memória RAM também é um fator importante, com um mínimo de 8GB para o uso de máquinas virtuais. No aspecto do armazenamento, o projeto não demanda um grande volume de dados e, portanto, é mais interessante investir em tecnologias de armazenamento de alta velocidade como o SSD.

Para o software, os simuladores de rede costumam ser executados em ambientes Linux e Unix. A execução em ambientes Windows, ainda que possível, apresenta maiores dificuldades e fontes de erros. Portanto, tanto para o desenvolvimento quanto para os testes e avaliação de desempenho da proposta, recomenda-se o uso de um sistema operacional Linux, sendo o Ubuntu 20.04 LTS uma escolha interessante tanto pelos custos envolvidos, quanto pela documentação disponível.

Vale ressaltar que o Ubuntu poderá ser executado diretamente pelo hardware do computador, ou ainda por meio de uma máquina virtual. Se escolhida a estratégia da máquina virtual, a solução da Oracle, Virtual Box, se apresenta como uma ferramenta satisfatória para a execução da tarefa, ainda que outras soluções possam ser utilizadas.

Nos aspectos de desenvolvimento do gerador e, considerando-se um ambiente de simulação como o ns-3, as linguagens de programação C++ e Python serão as protagonistas no desenvolvimento do projeto. Os artefatos produzidos no decorrer do projeto serão armazenados em plataformas de disponibilização de código como o GitHub e o GitLab, dependendo da fase do projeto.

Definido o ambiente de desenvolvimento e de simulação, passa-se a fase de aprofundamento do estudo do funcionamento do protocolo HTTP. Nesta fase, as bases teóricas apresentadas em textos clássicos, bem como leituras específicas para o desenvolvimento do gerador, possibilitarão o entendimento e reflexão crítica para o uso de propostas de geradores já existentes e como estas ferramentas podem ser relevantes para o projeto proposto.

Com o escopo definido, segue-se a fase de desenvolvimento do gerador, considerando-se as características e regras de desenvolvimento do ambiente de simulação.

Uma vez alcançado o escopo do projeto, passa-se à fase de avaliação do gerador proposto. Para essa fase, um cenário simples de estudo do protocolo HTTP será utilizado para averiguar o desempenho do módulo desenvolvido no decorrer do projeto.

A última fase é composta pela consolidação da documentação do módulo proposto, compartilhamento com a comunidade de pesquisadores e redação do artigo científico.
