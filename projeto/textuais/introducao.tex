% !TeX root = ../projeto.tex
\newpage
%\thispagestyle{empty}
\pagenumbering{arabic}

\section{INTRODUÇÃO}
\label{sec:introdução}
As redes de computadores têm se destacado cada vez mais como um setor fundamental em diversos aspectos da sociedade contemporânea. Seja no trabalho, no estudo, ou entretenimento, a conexão com a Internet se tornou parte do cotidiano das pessoas. Com a pandemia causada pelo Covid-19, esta constatação se tornou ainda mais avassaladora e muitos processos de conversão de serviços para o espaço digital foram acelerados \cite{Lavado2020} \cite{Johnson2021}.

Esta mudança de escala no uso das redes, seja no aumento do tráfego, bem como nas diferentes formas que as aplicações oferecem serviços para os usuários, traz consigo grandes desafios para a infraestrutura de rede e os atuais protocolos de comunicação.

As redes de comunicação por si só são ambientes complexos, frutos de um grande esforço de engenharia, controle e padrões estabelecidos. Nesta dinâmica de constante crescimento e oferta de novas aplicações, a inovação, revisão e proposição de novas soluções para os desafios das comunicações é um fator primordial para o sucesso deste sistema. Por outro lado, a concepção de elementos inovadores em um sistema tão grande e complexo pode ser algo impraticável quando consideramos aspectos de custos e tempo investidos nas pesquisas.

Neste contexto, os estudos realizados a partir de simuladores de rede oferecem uma boa contrapartida entre custos, tempo e resultados obtidos. O desenvolvimento de modelos matemáticos e simuladores para sistemas reais coloca-se como um importante campo de pesquisa. Simuladores desenvolvidos pela comunidade de pesquisadores, com a premissa do código aberto, apresentam características muito atrativas para o desenvolvimento de novas pesquisas, aprimoramento de protocolos e melhoria das atuais redes de comunicação.

Nesse sentido, o presente projeto de pesquisa propõe o estudo para a concepção de um gerador de tráfego para fluxos de dados característicos do protocolo HTTP (\textit{Hyper-Text Transfer Protocol})\nomenclature{HTTP}{Hyper-Text Transfer Protocol}. Dada a importância do protocolo HTTP nas comunicações, avalia-se a hipótese da relevância e viabilidade do desenvolvimento de um gerador de tráfego para o protocolo HTTP que fomente novos estudos e inovação na forma como este protocolo é utilizado pelos dispositivos da rede.

\subsection{TEMA}
\label{subsec:tema}
Dada a importância das redes de comunicações atuais e, considerando-se seu funcionamento baseado em protocolos, percebe-se a necessidade da constante análise do desempenho destes protocolos e estratégias de alocação de recursos em uma rede.

Neste sentido, estudos baseados em simulação tem sido uma ferramenta base para estas avaliações. Portanto, a construção de módulos para ferramentas de simulação se apresenta como um importante suporte para o fomento de pesquisas nessa área. 


\subsection{PROBLEMA}
\label{subsec:problema}
Considerando-se o amplo espectro de campos de pesquisa na área de Redes de Computadores e a importância da simulação na execução dessas atividades de pesquisa, percebe-se que grande parte destas iniciativas de estudos demandam um tráfego sintético para avaliar e validar hipóteses formuladas \cite{Cheng2013}.

A geração deste tráfego sintético deve ser amparada pela observação e registro do comportamento do tráfego em sistemas reais. Ao transferir este comportamento para um módulo gerador de tráfego em um ambiente de simulação, podem ser realizadas simplificações e abstrações com o objetivo de agilizar os estudos e focar nos aspectos mais relevantes.

O tráfego de uma rede de comunicação é composto por contribuições de diversos protocolos e perfis de uso dos usuários daquela rede. Nessa pluralidade de protocolos, o HTTP ganha destaque, uma vez que é utilizado tanto para a navegação web, quanto para transmissão de vídeos. A navegação web já foi considerada o tráfego dominante na Internet \cite{Pries2012} e as aplicações de vídeo têm projeção de representar 82\% de todo o tráfego da Internet em 2022 \cite{cisco-newsroom-vni-2017-2022}.

Estas estatísticas posicionam o tráfego HTTP como um importante elemento no cenário de geradores de tráfego, seja para estudos do próprio protocolo HTTP ou ainda como tráfego de entrada para outros tipos de estudos como, por exemplo, a alocação de recursos.


\subsection{OBJETIVO GERAL}
\label{subsec:objetivo-geral}
O objetivo geral deste projeto de pesquisa é desenvolver um gerador de tráfego para o protocolo HTTP, que permita tanto o estudo do protocolo em questão, como a utilização em estudos de outros aspectos de uma rede de comunicação, atuando como tráfego de entrada.

\subsection{OBJETIVOS ESPECÍFICOS}
\label{subsec:objetivos-específicos}
O presente projeto apresenta os seguintes objetivos específicos:
\begin{itemize}
	\item Investigar quais os modelos existentes para a caracterização do tráfego HTTP.
	\item Desenvolver um módulo para simulador de rede, com base nos modelos matemáticos desenvolvidos pelos pesquisadores da área.
	\item Testar o módulo em cenários de avaliação de desempenho em ambientes simulados.
	\item Documentar e disponibilizar o módulo para a comunidade científica.
\end{itemize}

\subsection{JUSTIFICATIVA}
\label{subsec:justificativa}
Considerando-se a importância das redes de comunicação no cenário atual, bem como a complexidade para estudos e processos de inovação dos protocolos que fazem parte destas redes, nota-se a relevância das ferramentas de simulação neste contexto de pesquisa.


Dessa forma, observa-se ainda que, pela relevância do tema, muitas alternativas são propostas pela comunidade de pesquisadores. Contudo, existe uma lacuna de módulos para este fim e que estejam vinculados a um ambiente de simulação já consolidado entre os pesquisadores. A integração do módulo gerador de tráfego dentro do ambiente de simulação facilita o uso por parte dos pesquisadores e fomenta o aprimoramento da ferramenta ao longo do tempo.


\subsection{DELIMITAÇÃO}
\label{subsec:delimitação}
Conforme descrito na seção \ref{subsec:problema}, o tráfego de uma rede de comunicação é a composição da atuação de diversas aplicações e diferentes comportamentos dos usuários. É um cenário de grande complexidade. Soma-se a isso o fato de que geradores de tráfego sintético devem ser baseados em modelos matemáticos que, por sua vez, são derivados de estudos empíricos de tráfego real. Portanto, propostas de implementação de geradores de tráfego podem ter diferentes escopos, de acordo com o rigor que se deseja aplicar, experiência do desenvolvedor e tempo para o desenvolvimento.

A partir desse contexto, é possível definir o escopo do projeto, ponderando-se aspectos como os objetivos específicos já apresentados e nível de exigência para um projeto de conclusão de curso. Portanto, no presente projeto propõe-se o seguinte escopo:

\begin{itemize}
	\item Dada a falta de publicações com modelos atualizados de tráfego real para o comportamento do HTTP em aplicações web, será adotado o modelo matemático apresentado em \cite{Pries2012}.
	\item Considerando-se a relevância e crescimento do uso do HTTP nas aplicações de vídeo, para este tipo de tráfego serão adotados os modelos matemáticos referendados em \cite{Navarro-Ortiz2020}.
	\item Considerando-se a experiência do estudante e tempo para a conclusão do projeto, nos aspectos técnicos do protocolo, optou-se pela implementação da versão HTTP 1.1, sem paralelismo e sem persistência na conexão, bem como aspectos do streaming adaptativo.
	\item Será realizada a implementação do cabeçalho do protocolo HTTP para possibilitar estudos do protocolo em questão.
	\item O ambiente de simulação será o responsável por oferecer a estrutura do restante da pilha de protocolos, como transferência confiável a partir do protocolo TCP, endereçamento e roteamento da camada de rede, etc. 
\end{itemize}

Em suma, o gerador de tráfego apresentará dois modelos de tráfego, modelo para o cabeçalho HTTP e cenários básicos para a aplicação do gerador em contextos de avaliação de protocolos e mecanismos de uma rede.

