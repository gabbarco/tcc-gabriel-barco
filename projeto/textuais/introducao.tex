% !TeX root = ../projeto.tex
\newpage
%\thispagestyle{empty}
\pagenumbering{arabic}

\section{INTRODUÇÃO}
\label{sec:introdução}
Os serviços de diretório têm sido cada vez mais utilizado em todas as
organizações do mundo, sejam faculdades, empresas, entre outras. Essa
grande demanda se deve ao fato de que em qualquer uma dessas é praticamente essencial centralizar a autenticação de usuários, com o objetivo de reduzir custos e facilitar a administração \cite{berbelini}. Com esse aumento em sua necessidade, vários serviços como: Active Directory (AD), OpenLDAP e Edirectory, vêm se destacando no mercado.

Estas ferramentas não se limitam apenas à autenticação de usuários, como também podem servir como suporte para a realização de atividades tais como nomeação, localização, segurança, entre outras relacionadas a gerir a infraestrutura de recursos nas organizações \cite{cruz2023}.

Esses serviços surgiram mediante à necessidade de se ter um único diretório que possibilitasse ao usuário acessar todos os recursos da empresa com apenas uma única senha, o que não era antes feito, assim, funcionários de uma empresa por exemplo, possuíam várias delas para acessarem diferentes setores. Como a própria Microsoft define, "o  Active  Directory é  um  serviço  de  diretório  que  armazena  informações sobre objetos em rede e disponibiliza essas informações a usuários e administradores de rede" \cite{microsoftlearn}.

Com sua popularização, se tornou indispensável para qualquer organização, logo,  não demorou para que fossem criados serviços de código aberto, isto é, totalmente gratuito e para uso geral, o que se refere ao caso do OpenLDAP. "Comumente o AD é confundido como o concorrente do LDAP, o que não é o caso, porquê ele não é nada mais que um exemplo de um serviço de diretório que suporta LDAP" \cite{bertolli}. O LDAP então, é o protocolo de serviços de diretório, ou seja, um meio para consultar os itens em qualquer um dos mesmos que o suportam.


\subsection{TEMA}
\label{subsec:tema}
O uso dos sistemas de diretórios é de suma importância para profissionais de TI, administradores de sistemas e gestores de empresas que desejam estabelecer um sistema eficiente de gerenciamento de acesso e recursos em um servidor.

A instalação e configuração adequadas do OpenLDAP em um ambiente ProxMox possibilitam um controle centralizado e seguro dos usuários, grupos e permissões de acesso, além de fornecer uma base sólida para o gerenciamento dos recursos disponíveis no servidor, como armazenamento, processamento e rede.


\subsection{PROBLEMA}
\label{subsec:problema}
Os serviços de diretório são uma realidade concreta em organizações com redes de grande porte. Contudo, em pequenas empresas, o uso não é tão disseminado, por vezes por desconhecimento de tal tecnologia ou mesmo falta de investimento em profissionais e equipamentos para operacionalizar essas funcionalidades. No segundo caso, esse problema se deve ao fato de que o sistema de diretório mais conhecido é o Microsoft Active Directory, que segundo a 6SENSE, "tem participação de mercado de 29,53\% no mercado de gerenciamento de identidade e acesso. O AD compete com 165 ferramentas concorrentes na categoria de gerenciamento de identidade e acesso" \cite{6sense}, esse que exige a compra de uma licença, quando se trata da administração remota em um servidor.

Esse fato abre espaço para a existência de ferramentas de código aberto, como o OpenLDAP, que será estudado nesse trabalho como uma alternativa ao AD, que se torna completamente desafiador para redes de pequeno porte, quando é vista a falta de recursos financeiros para a alocação desse tipo de sistema.


\subsection{OBJETIVO GERAL}
\label{subsec:objetivo-geral}
O objetivo geral deste artigo é instalar e configurar o serviço de diretórios "OpenLDAP" em um servidor "ProxMox", visando estabelecer um sistema eficiente de gerenciamento de acesso e recursos no servidor.

\subsection{OBJETIVOS ESPECÍFICOS}
\label{subsec:objetivos-específicos}
O presente projeto apresenta os seguintes objetivos específicos:
\begin{itemize}
    \item Instalar o ambiente virtual do ProxMox na máquina e rede destinadas para 
    a criação do servidor.
	\item Configurar o servidor ProxMox, garantindo um ambiente adequado para a implementação do serviço de diretórios.
    \item Pesquisar e compreender os conceitos fundamentais do OpenLDAP, incluindo sua estrutura, protocolos e recursos de segurança.
    \item Realizar a instalação do OpenLDAP em meio ao servidor já configurado.
    \item Configurar o OpenLDAP para autenticação de usuários e grupos no servidor ProxMox, estabelecendo políticas de acesso e permissões adequadas.
    \item Integrar o OpenLDAP com o sistema de gerenciamento de recursos do servidor ProxMox, permitindo o controle e monitoramento centralizado de recursos como armazenamento, processamento e rede.
    \item Realizar testes e validação da instalação e configuração do OpenLDAP, verificando a correta autenticação de usuários e a gestão eficiente dos recursos no servidor ProxMox.
    \item Documentar todo o processo de instalação e configuração do OpenLDAP no servidor ProxMox, fornecendo um guia passo a passo para auxiliar outros administradores de sistemas na implementação dessa solução.
\end{itemize}

\subsection{JUSTIFICATIVA}
\label{subsec:justificativa}
Esse trabalho foi pensado com a premissa de satisfazer uma necessidade do PET Computação IFTM Campus Ituiutaba, visto que com o crescimento no número de membros do programa, vêm crescendo cada vez mais a necessidade de se criar um gerenciamento do processo de autenticação nas máquinas. Considerando o que já foi relatado na seção que trata do problema alvo deste trabalho, o OpenLDAP se tornou uma alternativa a ser avaliada para essa tarefa. Espera-se que a implementação da solução proposta irá permitir aos participantes maior segurança no uso dos computadores e também irá possibilitar que possam ter logins de usuário com suas informações em qualquer máquina da rede.

Além disso, serão consideradas as necessidades do gerenciamento de recursos dentro do servidor. Visto que, ao catalogar e organizar as informações deles no diretório LDAP, é possível obter uma maior garantia da integridade e disponibilidade dos mesmos.

\subsection{DELIMITAÇÃO}
\label{subsec:delimitação}
Como visto anteriormente na seção \ref{subsec:problema}, o gerenciamento de acesso e recursos de um servidor, necessita de vários requisitos para ser implementado. Nesta seção, serão abordadas as limitações mais relevantes a serem consideradas na execução 
do projeto. 

Irão ser analisadas para a produção do escopo, as restrições que podem impactar a aquisição de hardware, licenças e soluções adicionais, bem como a necessidade de buscar alternativas econômicas para viabilizar a implementação. Além disso, discutiremos as limitações de escala, levando em consideração que o projeto é direcionado a redes de pequeno porte, com um número limitado de usuários e recursos a serem gerenciados.

Portanto, no presente projeto propõe-se as seguintes limitações:

\begin{itemize}
	\item Considerada a falta de recursos financeiros, não haverá uma comparação de eficiência entre o OpenLDAP e o Active Directory (ou outros serviços licenciados), se restrigindo apenas à análise do primeiro.
    \item Visto que o projeto é voltado para redes de pequeno porte, isso implica em um número limitado de usuários e recursos a serem gerenciados. As soluções propostas podem não ser escaláveis para redes de porte superior, requerendo configurações adicionais.
\end{itemize}

