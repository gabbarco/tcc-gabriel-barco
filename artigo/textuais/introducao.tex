% !TeX root = ../artigo.tex
\section{INTRODUÇÃO}
\pagenumbering{arabic}
As redes de computadores têm se destacado cada vez mais como um setor fundamental em diversos aspectos da sociedade contemporânea. Seja no trabalho, no estudo, ou entretenimento, a conexão com a Internet se tornou parte do cotidiano das pessoas. Com a pandemia causada pelo Covid-19, esta constatação se tornou ainda mais avassaladora e muitos processos de conversão de serviços para o espaço digital foram acelerados \cite{Lavado2020} \cite{Johnson2021}.

Esta mudança de escala no uso das redes, seja no aumento do tráfego, bem como nas diferentes formas que as aplicações oferecem serviços para os usuários, traz consigo grandes desafios para a infraestrutura de rede e os atuais protocolos de comunicação.

As redes de comunicação por si só são ambientes complexos, frutos de um grande esforço de engenharia, controle e padrões estabelecidos. Nesta dinâmica de constante crescimento e oferta de novas aplicações, a inovação, revisão e proposição de novas soluções para os desafios das comunicações é um fator primordial para o sucesso deste sistema. Por outro lado, a concepção de elementos inovadores em um sistema tão grande e complexo pode ser algo impraticável quando consideramos aspectos de custos e tempo investidos nas pesquisas.

Neste contexto, os estudos realizados a partir de simuladores de rede oferecem uma boa contrapartida entre custos, tempo e resultados obtidos. O desenvolvimento de modelos matemáticos e simuladores para sistemas reais coloca-se como um importante campo de pesquisa. Simuladores desenvolvidos pela comunidade de pesquisadores, com a premissa do código aberto, apresentam características muito atrativas para o desenvolvimento de novas pesquisas, aprimoramento de protocolos e melhoria das atuais redes de comunicação.

Nesse sentido, o presente projeto de pesquisa propõe o estudo para a concepção de um gerador de tráfego para fluxos de dados característicos do protocolo HTTP (\textit{Hyper-Text Transfer Protocol}). 