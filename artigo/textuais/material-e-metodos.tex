% !TeX root = ../artigo.tex
\section{MATERIAL E MÉTODOS}

Para o desenvolvimento do gerador proposto se faz necessário o uso intenso de ferramentas para o desenvolvimento de programas computacionais. 

Com relação ao hardware, utilizou-se um notebook com processador AMD Ryzen 5 e 8GB de memória RAM. No aspecto do armazenamento, o projeto não demandou um grande volume de dados e, portanto, o uso de um SSD com espaço de pelo menos 100GB é o bastante.

Para o software, os simuladores de rede costumam ser executados em ambientes Linux e Unix. A execução em ambientes Windows, ainda que possível, apresenta maiores dificuldades e fontes de erros. Portanto, tanto para o desenvolvimento quanto para os testes e avaliação de desempenho da proposta, utilizou-se o sistema operacional Linux Ubuntu 20.04 LTS com virtualização via Virtual Box.


Nos aspectos de desenvolvimento do gerador e, considerando-se um ambiente de simulação como o ns-3, as linguagens de programação C++ e Python serão as protagonistas no desenvolvimento do projeto. Os artefatos produzidos no decorrer do trabalho foram armazenados na plataforma de disponibilização de código GitLab, já empregada pelos desenvolvedores do ns-3.

